%%%%%%%%%%%%%%%%%%%%%%%%%%%%%%%%%%%%%%%%%%%%%%%%%%%%%%%%%%%%%%%%%%%%%%%%%%%%%%%%

\section{Haute disponibilité (HA)}

%%%%%%%%%%%%%%%%%%%%%%%%%%%%%%%%%%%%%%%%%%%%%%%%%%%%%%%%%%%%%%%%%%%%%%%%%%%%%%%%

\begin{frame}[fragile]{Solutions de HA}

   \begin{itemize}
      \item Il existe plusieurs solutions de HA.
      \item Certaines solutions intègrent nativement le partitionnement de données.
      \item Citus fait partie des solutions HA avec partitionnement de données
      \item L'objectif de cette formation est de permettre au client d'administrer ses bases de données avec le choix des outils qu'il a réalisé
      \item Le choix réalisé est \textbf{repmgr}
   \end{itemize}

\begin{toile}
\toileurl{https://docs.citusdata.com/en/v11.2/get\_started/what\_is\_citus.html}
\toileurl{https://repmgr.org/}
\end{toile}

\end{frame}

%%%%%%%%%%%%%%%%%%%%%%%%%%%%%%%%%%%%%%%%%%%%%%%%%%%%%%%%%%%%%%%%%%%%%%%%%%%%%%%%

\begin{frame}[fragile]{Présentation de \textbf{repmgr}}

   \begin{itemize}
      \item repmgr est une solution de gestion de:
         \begin{itemize}
            \item la réplication
            \item le switchover
            \item et le failover PostgreSQL
         \end{itemize}
      \item Elle supporte les versions PostgreSQL de 9.4 à 15
      \item Elle est portée par l'entreprise EDB
   \end{itemize}

\end{frame}

%%%%%%%%%%%%%%%%%%%%%%%%%%%%%%%%%%%%%%%%%%%%%%%%%%%%%%%%%%%%%%%%%%%%%%%%%%%%%%%%

\begin{frame}[fragile]{Caractéristiques avancées de \textbf{repmgr}}

   \begin{itemize}
      \item repmgr intègre les nouveautés introduites par PostgreSQL 9.3:
         \begin{itemize}
            \item la réplication en cascade
            \item la commutation de la ligne de temps (timeline switching)
            \item et la sauvegarde de base via la réplication
         \end{itemize}
      \item Il est disponible sous la licence GPLv3
      \item La dernière version disponible pendant la rédaction de ce document est la v5.3.3
   \end{itemize}

\end{frame}

%%%%%%%%%%%%%%%%%%%%%%%%%%%%%%%%%%%%%%%%%%%%%%%%%%%%%%%%%%%%%%%%%%%%%%%%%%%%%%%%

\section{Partitionnement des données}

%%%%%%%%%%%%%%%%%%%%%%%%%%%%%%%%%%%%%%%%%%%%%%%%%%%%%%%%%%%%%%%%%%%%%%%%%%%%%%%%

\begin{frame}[fragile]{Présentation des indexes B-Tree}

   \begin{itemize}
      \item WIP

   \end{itemize}

\begin{toile}
\toileurl{https://www.postgresql.org/docs/15/btree.html}
\toileurl{https://www.postgresql.org/docs/15/btree-implementation.html}
\end{toile}

\end{frame}

%%%%%%%%%%%%%%%%%%%%%%%%%%%%%%%%%%%%%%%%%%%%%%%%%%%%%%%%%%%%%%%%%%%%%%%%%%%%%%%%

\begin{frame}[fragile]{Tables partitionnées}

   \begin{itemize}
      \item PostgreSQL supporte le partitionnement de données.
      \item Le partitionnement PostgreSQL consiste à découper une table logique en plusieurs petits morceaux physiques
      \item Dans certaines situations, le partitionnement a plusieurs avantages:
      \begin{itemize}
         \item lorsque les données accédées sont localisées dans une partition ou un nombre réduit de partitions. A ce moment, la partie haute des indexes (B-Tree) qui correspond à ces données est entièrement en cache.
      \end{itemize}

   \end{itemize}

\begin{toile}
\toileurl{https://www.postgresql.org/docs/15/ddl-partitioning.html}
\end{toile}

\end{frame}

