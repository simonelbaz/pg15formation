%%%%%%%%%%%%%%%%%%%%%%%%%%%%%%%%%%%%%%%%%%%%%%%%%%%%%%%%%%%%%%%%%%%%%%%%%%%%%%%

\section{Jeu de caractères}

%%%%%%%%%%%%%%%%%%%%%%%%%%%%%%%%%%%%%%%%%%%%%%%%%%%%%%%%%%%%%%%%%%%%%%%%%%%%%%%%

\begin{frame}[fragile]{Support de la localisation}

\begin{itemize}
   \item PostgreSQL supporte 2 outils pour la localisation:
   \begin{itemize}
      \item Utilisation des paramètres locaux du système
      \item Fourniture de jeux de caractères pour le stockage dans tous les langages. Les jeux de caractères permettent la traduction d'un alphabet à un autre entre le client et le serveur
   \end{itemize}
\end{itemize}

\begin{toile}
\toileurl{https://www.postgresql.org/docs/15/charset.html}
\end{toile}

\end{frame}

%%%%%%%%%%%%%%%%%%%%%%%%%%%%%%%%%%%%%%%%%%%%%%%%%%%%%%%%%%%%%%%%%%%%%%%%%%%%%%%%

\begin{frame}[fragile]{Utilisation des paramètres locaux du système}

\begin{itemize}
   \item Les paramètres locaux du système apportent les fonctionnalités suivantes:
   \begin{itemize}
      \item le tri des chaînes et des caractères propres au langage local
      \item le formatage des nombres
      \item la traduction des messages
      \item \ldots
   \end{itemize}
\end{itemize}

\end{frame}

%%%%%%%%%%%%%%%%%%%%%%%%%%%%%%%%%%%%%%%%%%%%%%%%%%%%%%%%%%%%%%%%%%%%%%%%%%%%%%%%

\begin{frame}[fragile]{Support de la \textbf{locale}}

\begin{itemize}
   \item Le support d'une \textbf{locale} signifie que l'application respecte les préférences culturelles pour un alphabet, le tri et le format des nombres
   \item PostgreSQL utilise les standards \textit{ISO C} et \textit{POSIX locale} fournis par l'OS du serveur
   \item Le support de la locale pour un serveur de base de données pendant la phase d'\textbf{initdb}
   \item \textbf{initdb} initialise le serveur avec la locale de son environnement d'exécution
\end{itemize}

\end{frame}

%%%%%%%%%%%%%%%%%%%%%%%%%%%%%%%%%%%%%%%%%%%%%%%%%%%%%%%%%%%%%%%%%%%%%%%%%%%%%%%%

\begin{frame}[fragile]{Récupération de la \textbf{locale} d'un environnement}

\begin{itemize}
   \item La commande ci-dessous permet de récupérer la locale de l'environnement:
\begin{tiny}
\begin{Verbatim}[commandchars=\&\{\}]
[linagora@localhost ~]$ locale
LANG=fr\_FR.UTF-8
LC\_CTYPE="fr\_FR.UTF-8"
LC\_NUMERIC="fr\_FR.UTF-8"
LC\_TIME="fr\_FR.UTF-8"
LC\_COLLATE="fr\_FR.UTF-8"
LC\_MONETARY="fr\_FR.UTF-8"
LC\_MESSAGES="fr\_FR.UTF-8"
LC\_PAPER="fr\_FR.UTF-8"
LC\_NAME="fr\_FR.UTF-8"
LC\_ADDRESS="fr\_FR.UTF-8"
LC\_TELEPHONE="fr\_FR.UTF-8"
LC\_MEASUREMENT="fr\_FR.UTF-8"
LC\_IDENTIFICATION="fr\_FR.UTF-8"
LC\_ALL=
\end{Verbatim}
\end{tiny}
\end{itemize}

\end{frame}

%%%%%%%%%%%%%%%%%%%%%%%%%%%%%%%%%%%%%%%%%%%%%%%%%%%%%%%%%%%%%%%%%%%%%%%%%%%%%%%%

\begin{frame}[fragile]{Choix de la locale à l'initdb}

\begin{itemize}
   \item Il est possible de faire le choix explicite de la localte à l'initdb:
\begin{tiny}
\begin{Verbatim}[commandchars=\&\{\}]
initdb --locale=fr\_CA
\end{Verbatim}
\end{tiny}
   \item S'il existe plusieurs jeux de caractères pour une lcale, il est également possible d'ajouter le jeu de caractères:
\begin{tiny}
\begin{Verbatim}[commandchars=\&\{\}]
initdb --locale=fr\_CA.UTF-8
\end{Verbatim}
\end{tiny}
   \item Le format est donc: \textit{langage\_territoire.encodage}
\end{itemize}

\end{frame}

%%%%%%%%%%%%%%%%%%%%%%%%%%%%%%%%%%%%%%%%%%%%%%%%%%%%%%%%%%%%%%%%%%%%%%%%%%%%%%%%

\begin{frame}[fragile]{Locales disponibles dans un environnement}

\begin{itemize}
   \item Pour obtenir la liste des locales disponibles dans un environnement, la commande suivante peut être utilisée:
\begin{tiny}
\begin{Verbatim}[commandchars=\&\{\}]
locale -a
\end{Verbatim}
\end{tiny}
   \item La signification des paramètres \textbf{LC\_*} est la suivante:
   \begin{itemize}
      \item LC\_COLLATE: tri des chaînes de caractères
      \item LC\_CTYPE: classification des caractères (qu'est ce qu'une lettre ? quelle est son équivalent en majuscule ?)
      \item LC\_MESSAGES: langage des messages
      \item LC\_MONETARY: formatage des monnaies
      \item LC\_NUMERIC: formatage des nombres
      \item LC\_TIME: formatage des dates et des durées
   \end{itemize}
   \item Il est possible de modifier ces paramètres en tant que variables d'environnement
\end{itemize}

\end{frame}

%%%%%%%%%%%%%%%%%%%%%%%%%%%%%%%%%%%%%%%%%%%%%%%%%%%%%%%%%%%%%%%%%%%%%%%%%%%%%%%%

\begin{frame}[fragile]{Modification des variables \textbf{LC\_*} à l'\textbf{initdb}}

\begin{itemize}
   \item Pour surcharger la définition des variables d'environnement \textbf{LC\_*} à l'initdb, celui-ci peut être lancé de la manière suivante:
\begin{tiny}
\begin{Verbatim}[commandchars=\&\{\}]
initdb --locale=fr\_CA --lc-monetary=en\_US
\end{Verbatim}
\end{tiny}
   \item Si le DBA souhaite que le serveur ne supporte pas de locale, il est possible d'utiliser la locale spéciale \textbf{C} ou son équivalent \textbf{POSIX}
\end{itemize}

\end{frame}

%%%%%%%%%%%%%%%%%%%%%%%%%%%%%%%%%%%%%%%%%%%%%%%%%%%%%%%%%%%%%%%%%%%%%%%%%%%%%%%%

\begin{frame}[fragile]{Changement des valeurs \textbf{LC\_*} après l'initdb}

\begin{itemize}
   \item Certaines 
\end{itemize}

\end{frame}

%%%%%%%%%%%%%%%%%%%%%%%%%%%%%%%%%%%%%%%%%%%%%%%%%%%%%%%%%%%%%%%%%%%%%%%%%%%%%%%%

\begin{frame}[fragile]{La collation}

\begin{itemize}
   \item \textbf{Définition:} La collation permet de définir le tri et la classification des caractères des données par colonnes ou par opération
   \item Cela permet de contourner la restriction que les paramètres \textbf{LC\_COLLATE} et \textbf{LC\_CTYPE} ne peuvent pas être modifiés après la création d'une base de données
\end{itemize}

\begin{toile}
\toileurl{https://www.postgresql.org/docs/15/collation.html}
\end{toile}

\end{frame}

%%%%%%%%%%%%%%%%%%%%%%%%%%%%%%%%%%%%%%%%%%%%%%%%%%%%%%%%%%%%%%%%%%%%%%%%%%%%%%%%

\begin{frame}[fragile]{Changement de collation}

\begin{itemize}
   \item WIP
\end{itemize}

\begin{toile}
\toileurl{https://www.postgresql.org/docs/15/sql-altercollation.html\#SQL-ALTERCOLLATION-NOTES}
\end{toile}

\end{frame}

