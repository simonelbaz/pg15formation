%%%%%%%%%%%%%%%%%%%%%%%%%%%%%%%%%%%%%%%%%%%%%%%%%%%%%%%%%%%%%%%%%%%%%%%%%%%%%%%%

\section{Sécurisation de la base PostgreSQL}

%%%%%%%%%%%%%%%%%%%%%%%%%%%%%%%%%%%%%%%%%%%%%%%%%%%%%%%%%%%%%%%%%%%%%%%%%%%%%%%%

\begin{frame}[fragile]{Chiffrement dans PostgreSQL}

   \begin{itemize}
      \item Il existe différents niveaux de chiffrement dans PostgreSQL
      \begin{itemize}
         \item Chiffrement du mot de passe. Paramètre \textbf{password\_encryption}
         \item Chiffrement de colonnes spécifiques. Le module pgcrypto permet de chiffrer/déchiffrer des colonnes.
         \begin{itemize}
            \item Le client fournit la clé de déchiffrement au serveur qui lui renvoie la donnée en clair
            \item La donnée en clair et la clef de déchiffrement résident temporairement sur le serveur
            \item Cet instant court d'exposition de la donnée et de la clef permettraient à un administrateur d'intercepter ces données
         \end{itemize}
      \end{itemize}
   \end{itemize}

\begin{toile}
\toileurl{https://www.postgresql.org/docs/15/encryption-options.html}
\toileurl{https://www.postgresql.org/docs/15/pgcrypto.html}
\end{toile}

\end{frame}

%%%%%%%%%%%%%%%%%%%%%%%%%%%%%%%%%%%%%%%%%%%%%%%%%%%%%%%%%%%%%%%%%%%%%%%%%%%%%%%%

\begin{frame}[fragile]{Chiffrement dans PostgreSQL}

   \begin{itemize}
         \item Chiffrement de partitions disque. Le chiffrement peut être réalisé au niveau du filesystem ou au niveau des blocs de données
         \begin{itemize}
            \item Les modules eCryptfs ou EncFS sur Linux permettent de chiffrer un filesystem
            \item Le chiffrement au niveau du bloc ou du disque est réalisé par \textbf{dm-crypt} + \textbf{LUKS}
            \item Ces mécanismes protègent de la lecture de données non chiffrées ou en cas de vol de disque
            \item Lorsque le disque est monté pour lecture en clair, il devient possible pour un attaquant d'accéder aux données
            \item Le montage du disque est toutefois protégé par une clef
         \end{itemize}
         \item Chiffrement des flux réseau. Les connexions SSL chiffrent les requêtes, les mots de passe et les données échangés entre le serveur et le client
         \begin{itemize}
            \item Le fichier \textbf{pg\_hba.conf} spécifie les hôtes qui discutent en clair (\textit{host}) et ceux qui discutent en chiffré (\textit{hostssl})
            \item Il est également possible d'établir des tunnels SSH pour discuter avec le serveur
         \end{itemize}
   \end{itemize}

\end{frame}

%%%%%%%%%%%%%%%%%%%%%%%%%%%%%%%%%%%%%%%%%%%%%%%%%%%%%%%%%%%%%%%%%%%%%%%%%%%%%%%%

\begin{frame}[fragile]{Authentification par certificats}

   \begin{itemize}
         \item Il est possible de déployer des certificats SSL au niveau serveur et client
         \item Cela nécessite du paramétrage mais fournit une identification mutuelle plus forte que le mot de passe
         \item Cela permet de certifier que le serveur est bien la machine accédée et empêche donc la récupération de mot de passe
         \item L'authentification par certificat prévient également les attaques de type \textit{man in the middle}
   \end{itemize}

\begin{toile}
\toileurl{https://www.postgresql.org/docs/15/encryption-options.html}
\end{toile}

\end{frame}

%%%%%%%%%%%%%%%%%%%%%%%%%%%%%%%%%%%%%%%%%%%%%%%%%%%%%%%%%%%%%%%%%%%%%%%%%%%%%%%%

\begin{frame}[fragile]{Authentification client - \textbf{pg\_hba.conf}}

   \begin{itemize}
      \item PostgreSQL propose plusieurs méthodes d'authentification basées sur l'adresse IP du client, le nom de la base et l'utilisateur
      \item HBA signifie \textit{Host Base Authentication}
      \item pg\_hba.conf est déployé par la phase initdb
      \item Il est possible de paramétrer sa localisation avec le paramètre \text{hba\_file}
      \item Le fichier est composé de plusieurs enregistrements
      \item Chaque enregistrement est un ensemble de champs séparés par un espace ou une tabulation
      \item Chaque champ peut inclure un espace s'il est entouré de double-quotes
      \item Chaque champ encadré par une quote perd sa signification spéciale
      \item Le caractère \textbackslash en fin de ligne prolonge la ligne à la ligne suivante
   \end{itemize}

\begin{toile}
\toileurl{https://www.postgresql.org/docs/15/client-authentication.html}
\toileurl{https://www.postgresql.org/docs/15/auth-pg-hba-conf.html}
\end{toile}

\end{frame}

%%%%%%%%%%%%%%%%%%%%%%%%%%%%%%%%%%%%%%%%%%%%%%%%%%%%%%%%%%%%%%%%%%%%%%%%%%%%%%%%

\begin{frame}[fragile]{Format de \textbf{pg\_hba.conf}}

   \begin{itemize}
      \item Voici le format possible des enregistrements de pg\_hba.conf:
\begin{tiny}
\begin{Verbatim}[commandchars=\\\{\}]
local         database  user  auth-method [auth-options]
host          database  user  address     auth-method  [auth-options]
hostssl       database  user  address     auth-method  [auth-options]
hostnossl     database  user  address     auth-method  [auth-options]
hostgssenc    database  user  address     auth-method  [auth-options]
hostnogssenc  database  user  address     auth-method  [auth-options]
host          database  user  IP-address  IP-mask      auth-method  [auth-options]
hostssl       database  user  IP-address  IP-mask      auth-method  [auth-options]
hostnossl     database  user  IP-address  IP-mask      auth-method  [auth-options]
hostgssenc    database  user  IP-address  IP-mask      auth-method  [auth-options]
hostnogssenc  database  user  IP-address  IP-mask      auth-method  [auth-options]
\end{Verbatim}
\end{tiny}
   \end{itemize}

\end{frame}

%%%%%%%%%%%%%%%%%%%%%%%%%%%%%%%%%%%%%%%%%%%%%%%%%%%%%%%%%%%%%%%%%%%%%%%%%%%%%%%%

\begin{frame}[fragile]{Types de connexion}

   \begin{itemize}
      \item local: Utilisation des sockets Unix pour accéder à la base de données
      \item host: Utilisation du protocole TCP pour des accès en clair ou SSL
      \item hostssl: Utilisation obligatoire de SSL pour le client
      \item hostnossl: SSL désactivé
   \end{itemize}

\end{frame}

%%%%%%%%%%%%%%%%%%%%%%%%%%%%%%%%%%%%%%%%%%%%%%%%%%%%%%%%%%%%%%%%%%%%%%%%%%%%%%%%

\begin{frame}[fragile]{Le champ bases de données}

   \begin{itemize}
      \item all: signifie toutes les bases de données
      \item sameuser: l'enregistrement correspond si la base de données a le même nom que l'utilisateur
      \item samerole: l'enregistrement correspond si l'utilisateur a le même rôle que le nom de la base
      \item replication: l'enregistrement correspond si une connexion pour une réplication physique est initialisée
      \item \textbf{Remarque}: Cela n'inclut pas les réplications logiques
      \item Un ou plusieurs noms de base séparés par une virgule peuvent être précisés
      \item Le nom d'un fichier précédé de '@' peut être précisé
   \end{itemize}

\end{frame}

%%%%%%%%%%%%%%%%%%%%%%%%%%%%%%%%%%%%%%%%%%%%%%%%%%%%%%%%%%%%%%%%%%%%%%%%%%%%%%%%

\begin{frame}[fragile]{Le champ utilisateur}

   \begin{itemize}
      \item all: signifie tous les utilisateurs
      \item Cela peut être le nom d'un utilisateur ou d'un rôle
      \item Si le champ est précédé d'un caractère '+', cela signifie tous les rôles faisant partie de ce rôle
      \item Un ou plusieurs noms d'utilisateurs séparés par une virgule peuvent être précisés
      \item Le nom d'un fichier précédé de '@' peut être précisé
   \end{itemize}

\end{frame}

%%%%%%%%%%%%%%%%%%%%%%%%%%%%%%%%%%%%%%%%%%%%%%%%%%%%%%%%%%%%%%%%%%%%%%%%%%%%%%%%

\begin{frame}[fragile]{Le champ \textbf{address}}

   \begin{itemize}
      \item all: signifie tous les hôtes
      \item samehost: toutes les adresses IP du serveur
      \item samenet: même sous-réseau
      \item un nom de machine (résolution DNS et résolution inverse sont appliquées))
      \item si le nom est précédé d'un '.', tous les serveurs du domaine correspondent
   \end{itemize}

\end{frame}

%%%%%%%%%%%%%%%%%%%%%%%%%%%%%%%%%%%%%%%%%%%%%%%%%%%%%%%%%%%%%%%%%%%%%%%%%%%%%%%%

\begin{frame}[fragile]{Le champ \textbf{auth-meth}}

   Les principals valeurs sont:
   \begin{itemize}
      \item trust: accepte inconditionnellement
      \item reject: rejette inconditionnellement
      \item peer: l'utilisateur du client (utilisateur système) correspond au nom de la base. Valide seulement pour le type local
      \item ident: l'utilisateur du client (utilisateur système) correspond au nom de la base. Valide seulement pour le type TCP (host)
   \end{itemize}

\end{frame}

%%%%%%%%%%%%%%%%%%%%%%%%%%%%%%%%%%%%%%%%%%%%%%%%%%%%%%%%%%%%%%%%%%%%%%%%%%%%%%%%

\begin{frame}[fragile]{La vue \textbf{pg\_hba\_file\_rules}}

   \begin{itemize}
      \item Cette vue est utile car elle permet de tester des changements de pg\_hba.conf
      \item Elle est accessible uniquement par les super-utilisateurs
      \item Elle permet également de diagnostiquer les erreurs de syntaxe
      \item Elle reflète ce qui est actuellement dans pg\_hba.conf et non ce qui a été chargé par le serveur
      \item En cas d'erreur de syntaxe, la colonne \textbf{error} est renseignée
   \end{itemize}
   
\begin{toile}
\toileurl{https://www.postgresql.org/docs/15/view-pg-hba-file-rules.html}
\end{toile}

\end{frame}

%%%%%%%%%%%%%%%%%%%%%%%%%%%%%%%%%%%%%%%%%%%%%%%%%%%%%%%%%%%%%%%%%%%%%%%%%%%%%%%%

\begin{frame}[fragile]{Les rôles}

   \begin{itemize}
      \item Un rôle peut être soit un utilisateur de la base soit un groupe d'utilisateurs soit les deux
      \item Cela dépend comment il a été créé
      \item Les rôles peuvent posséder des objets (table, fonction, \ldots)
      \item Ils peuvent assigner des droits sur ces objets à d'autres rôles
      \item Un rôle peut appartenir à un autre rôle
      \item Le catalogue système \textbf{pg\_roles} liste les rôles
   \end{itemize}

\begin{toile}
\toileurl{https://www.postgresql.org/docs/15/user-manag.html}
\end{toile}

\end{frame}

%%%%%%%%%%%%%%%%%%%%%%%%%%%%%%%%%%%%%%%%%%%%%%%%%%%%%%%%%%%%%%%%%%%%%%%%%%%%%%%%

\begin{frame}[fragile]{La variable \textbf{search\_path}}

   \begin{itemize}
      \item Cette variable spécifie l'ordre des schémas dans lesquels un objet est recherché lorsque celui-ci n'est pas préfixé par un nom de schéma
      \item Lorsque des objets ont le même nom dans 2 schémas différents, la priorité est donné à celui qui est trouvé le premier à partir du \textbf{search\_path}
      \item Un objet qui ne figure pas dans les schémas mentionnés par cette variable doit être précédé du nom du schéma
      \item Si l'un des schémas a pour valeur \$user, il est remplacé par le nom de l'utilisateur courant (\textbf{CURRENT\_USER})
      \item Pour accéder à un schéma, l'utilisateur doit avoir la permission \textbf{USAGE}
   \end{itemize}

\begin{toile}
\toileurl{https://www.postgresql.org/docs/15/runtime-config-client.html\#GUC-SEARCH-PATH}
\end{toile}

\end{frame}

%%%%%%%%%%%%%%%%%%%%%%%%%%%%%%%%%%%%%%%%%%%%%%%%%%%%%%%%%%%%%%%%%%%%%%%%%%%%%%%%

\begin{frame}[fragile]{Politiques de sécurité des lignes - Row Security Policies (RSP)}

   \begin{itemize}
      \item En plus de l'accès aux tables fourni par la commande \textbf{GRANT}, il est possible de préciser également les droits d'accès aux lignes d'une table
      \item Elle est activée par la clause \textbf{ENABLE ROW LEVEL SECURITY} de la commande \textbf{ALTER TABLE}
      \item A ce moment, la sélection ou la modification sont autorisés par une RSP
   \end{itemize}

\begin{toile}
\toileurl{https://www.postgresql.org/docs/current/ddl-rowsecurity.html}
\toileurl{https://www.postgresql.org/docs/current/sql-grant.html}
\end{toile}

\end{frame}

%%%%%%%%%%%%%%%%%%%%%%%%%%%%%%%%%%%%%%%%%%%%%%%%%%%%%%%%%%%%%%%%%%%%%%%%%%%%%%%%

\begin{frame}[fragile]{Exercice - Déploiement du chiffrement SSL entre le client et le serveur}

   \begin{itemize}
      \item Créer et déployer les certificats pour mettre en place un flux chiffré
   \end{itemize}

\begin{toile}
\toileurl{https://www.postgresql.org/docs/15/ssl-tcp.html}
\end{toile}

\end{frame}

