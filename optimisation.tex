%%%%%%%%%%%%%%%%%%%%%%%%%%%%%%%%%%%%%%%%%%%%%%%%%%%%%%%%%%%%%%%%%%%%%%%%%%%%%%%%

\section{Optimisation}

%%%%%%%%%%%%%%%%%%%%%%%%%%%%%%%%%%%%%%%%%%%%%%%%%%%%%%%%%%%%%%%%%%%%%%%%%%%%%%%%

\begin{frame}[fragile]{Les indexes dans PostgreSQL}

   \begin{itemize}
      \item B-Tree Index - très utile pour la recherche d'une valeur ou le scan d'une plage de valeur. Utilisé également pour les expressions régulières (pattern matching)
      \item Hash Index - très efficace pour recherche de valeurs égales
      \item Generalized Search Tree (GiST) - est une catégorie d'index offrant une architecture dans laquelle plusieurs stratégies d'indexes peuvent être implémentées. En fonction de ces stratégies, des opérateurs (\textit{operator class}) sont définies.
      \item Par exemple, la distribution standard de PostgreSQL inclut des opérateurs de classe GisT pour les types de données à 2 dimensions
      \item Les indexes GiST sont aussi capables d'optimiser les recherches de type \textbf{"voisinage le plus proche"}
   \end{itemize}

\begin{toile}
\toileurl{https://www.postgresql.org/docs/15/indexes-types.html}
\end{toile}

\end{frame}

%%%%%%%%%%%%%%%%%%%%%%%%%%%%%%%%%%%%%%%%%%%%%%%%%%%%%%%%%%%%%%%%%%%%%%%%%%%%%%%%

\begin{frame}[fragile]{Les indexes dans PostgreSQL}


   \begin{itemize}
      \item Space Partitioned GiST (SP-GiST) - similar au GiST, cette classe d'index supporte des structures de données non équilibrées
      \item Generalized Inverted Index (GIN) - utile pour indexer des valeurs de type multi-composants comme des tableaux. et tester la présence d'un élément
      \item Tout comme la famille GiST, la famille GIN supporte des stratégies d'indexation personnalisables par l'utilisateur. Les opérateurs de la cette classe d'index dépendent de la stratégie d'indexation.
   \end{itemize}

\end{frame}

%%%%%%%%%%%%%%%%%%%%%%%%%%%%%%%%%%%%%%%%%%%%%%%%%%%%%%%%%%%%%%%%%%%%%%%%%%%%%%%%

\begin{frame}[fragile]{Les indexes dans PostgreSQL}


   \begin{itemize}
      \item Block Range Index (BRIN) - ce type d'index stocke des résumés d'information pour des blocs contigus et physiques de données. Ce type d'index est bien adapté aux données corrélées avec l'ordre physique de stockage
      \item Tout comme la famille GiST, la famille BRIN supporte des stratégies d'indexation personnalisables par l'utilisateur. Les opérateurs de la cette classe d'index dépendent de la stratégie d'indexation.
      \item L'index de type BRIN contient le min et le max de chaque bloc de données linéaire
   \end{itemize}

\end{frame}

%%%%%%%%%%%%%%%%%%%%%%%%%%%%%%%%%%%%%%%%%%%%%%%%%%%%%%%%%%%%%%%%%%%%%%%%%%%%%%%%

\begin{frame}[fragile]{Pistes d'optimisation}

   \begin{itemize}
      \item La quantité de \textbf{shared\_buffers}: elle peut occuper 1/4 de la RAM disponible
      \item \textbf{work\_mem} est la quantité de RAM allouée à chaque process de traitement d'une session
      \item \textbf{effective\_cache\_size} est la quantité de RAM que le planificateur de requête a à sa disposition. En cas de valeur importante, il va privilégier l'utilisation des indexes. Pour une valeur plus faible, il privilégie les scans séquentiels.
      \item \textbf{effective\_io\_concurrency} est le nombre d'opération I/O que le serveur PostgreSQL peut exécuter en parallle sur un disque
      \item activation des Huge Pages
   \end{itemize}

\begin{toile}
\toileurl{https://www.postgresql.org/docs/15/runtime-config-query.html}
\toileurl{https://www.postgresql.org/docs/15/runtime-config-resource.html\#RUNTIME-CONFIG-RESOURCE-ASYNC-BEHAVIOR}
\toileurl{https://smartsla.08000linux.com/requests/1009}
\end{toile}

\end{frame}

%%%%%%%%%%%%%%%%%%%%%%%%%%%%%%%%%%%%%%%%%%%%%%%%%%%%%%%%%%%%%%%%%%%%%%%%%%%%%%%%

\begin{frame}[fragile]{Activation des Huge Pages du kernel}

   \begin{itemize}
      \item L'utilisation des Huge Pages Kernel permet d'allouer des pages de RAM plus imortantes au process PostgreSQL
      \item Il minimise les appels Kernel pour l'allocation de RAM
   \end{itemize}

   Pour vérifier que les Huge Pages sont activées, la commande suivante peut être utilisée:

\begin{tiny}
\begin{Verbatim}[commandchars=\\\{\}]
# cat /proc/meminfo | grep Huge
AnonHugePages:  247808 kB
HugePages_Total:  19781
HugePages_Free:  16193
HugePages_Rsvd:   5193
HugePages_Surp:    0
Hugepagesize:    2048 kB
\end{Verbatim}
\end{tiny}

\begin{toile}
\toileurl{https://www.kernel.org/doc/html/latest/admin-guide/mm/hugetlbpage.html}
\end{toile}

\end{frame}

%%%%%%%%%%%%%%%%%%%%%%%%%%%%%%%%%%%%%%%%%%%%%%%%%%%%%%%%%%%%%%%%%%%%%%%%%%%%%%%%

\begin{frame}[fragile]{EXPLAIN PLAN}

   TODO
   WIP

\begin{toile}
\toileurl{XXXX}
\end{toile}

\end{frame}

%%%%%%%%%%%%%%%%%%%%%%%%%%%%%%%%%%%%%%%%%%%%%%%%%%%%%%%%%%%%%%%%%%%%%%%%%%%%%%%%

\begin{frame}[fragile]{Index only scans}
   \label{indexonlyscan}

   \begin{itemize}
      \item Les index dans PostgreSQL sont dits secondaires car l'index et la donnée indexée sont stockés à 2 endroits différents
      \item Les données de la table sont stockées dans une zone appelée \textbf{heap}
      \item Cela signifie que lors d'un scan d'un index ordinaire, la récupération d'une ligne passe par la récupération d'information dans l'index puis dans la heap
      \item De plus, les données d'un index correspondant à une clause WHERE sont proches dans l'index. Les données qui leur correspondent peuvent être éparpillées dans la heap
      \item Le scan d'un index peut donc causer des accès random nombreux, qui peuvent se révéler lents dans le cas de certains médias à rotation.
      \item Pour résoudre ce problème de performances, PostgreSQL supporte les index-only scans. 
      \item Les index only scans renvoient les données indexées depuis l'index sans accéder à la heap.
   \end{itemize}

\begin{toile}
\toileurl{https://www.postgresql.org/docs/15/indexes-index-only-scans.html}
\end{toile}

\end{frame}

%%%%%%%%%%%%%%%%%%%%%%%%%%%%%%%%%%%%%%%%%%%%%%%%%%%%%%%%%%%%%%%%%%%%%%%%%%%%%%%%

\begin{frame}[fragile]{Index only scans}

   \begin{itemize}
      \item L'idée est de stocker la donnée indexée avec l'entrée dans l'index
      \item Il y a 2 restrictions à l'application de cet algorithme:
      \begin{itemize}
         \item Le type de l'index doit supporter les index only scans. Les index B-Tree supportent cet algorithme. GiST et SP-GisT le supportent pour certains opérateurs de classe mais pas pour d'autres.
         \item Les autres types d'index ne supportent pas les index only scans.
         \item La requête ne doit référencer que les colonnes de l'index
      \end{itemize}
      \item Exemple d'une table ayant les colonnes x, y et z.
      \item Les requêtes pouvant faire l'usage des index only scans sont:
\begin{tiny}
\begin{Verbatim}[commandchars=\\\{\}]
SELECT x, y FROM tab WHERE x = 'key';
SELECT x FROM tab WHERE x = 'key' AND y < 42;
\end{Verbatim}
\end{tiny}
   \end{itemize}

\end{frame}

%%%%%%%%%%%%%%%%%%%%%%%%%%%%%%%%%%%%%%%%%%%%%%%%%%%%%%%%%%%%%%%%%%%%%%%%%%%%%%%%

\begin{frame}[fragile]{Restrictions sur les index only scans}

   \begin{itemize}
      \item Les requêtes suivantes ne peuvent pas faire l'usage des index only scans:
\begin{tiny}
\begin{Verbatim}[commandchars=\\\{\}]
SELECT x, z FROM tab WHERE x = 'key';
SELECT x FROM tab WHERE x = 'key' AND z < 42;
\end{Verbatim}
\end{tiny}
   \end{itemize}

\end{frame}

%%%%%%%%%%%%%%%%%%%%%%%%%%%%%%%%%%%%%%%%%%%%%%%%%%%%%%%%%%%%%%%%%%%%%%%%%%%%%%%%

\begin{frame}[fragile]{La commande CLUSTER}

   TODO
   WIP

\begin{toile}
\toileurl{https://www.postgresql.org/docs/15/sql-cluster.html}
\end{toile}

\end{frame}

%%%%%%%%%%%%%%%%%%%%%%%%%%%%%%%%%%%%%%%%%%%%%%%%%%%%%%%%%%%%%%%%%%%%%%%%%%%%%%%%

\begin{frame}[fragile]{Utilisation des statistiques par le planificateur de requêtes}

   TODO
   WIP

\begin{toile}
\toileurl{https://www.postgresql.org/docs/15/planner-stats.html}
\end{toile}

\end{frame}

%%%%%%%%%%%%%%%%%%%%%%%%%%%%%%%%%%%%%%%%%%%%%%%%%%%%%%%%%%%%%%%%%%%%%%%%%%%%%%%%

\begin{frame}[fragile]{Statistiques accumulées pendant le service - Run-time statistics}

   TODO
   WIP

\begin{toile}
\toileurl{https://www.postgresql.org/docs/15/runtime-config-statistics.html}
\end{toile}

\end{frame}

%%%%%%%%%%%%%%%%%%%%%%%%%%%%%%%%%%%%%%%%%%%%%%%%%%%%%%%%%%%%%%%%%%%%%%%%%%%%%%%%

\begin{frame}[fragile]{Statistiques étendues}

   TODO
   WIP

\begin{toile}
\toileurl{https://www.postgresql.org/docs/current/planner-stats.html\#PLANNER-STATS-EXTENDED}
\toileurl{https://www.postgresql.org/docs/current/multivariate-statistics-examples.html}
\end{toile}

\end{frame}

%%%%%%%%%%%%%%%%%%%%%%%%%%%%%%%%%%%%%%%%%%%%%%%%%%%%%%%%%%%%%%%%%%%%%%%%%%%%%%%%

\begin{frame}[fragile]{Requêtes parallèles}

   TODO
   WIP

\begin{toile}
\toileurl{https://www.postgresql.org/docs/15/parallel-query.html}
\end{toile}

\end{frame}

%%%%%%%%%%%%%%%%%%%%%%%%%%%%%%%%%%%%%%%%%%%%%%%%%%%%%%%%%%%%%%%%%%%%%%%%%%%%%%%%

\begin{frame}[fragile]{Amélioration des performances}

   TODO
   WIP

\begin{toile}
\toileurl{https://www.postgresql.org/docs/15/performance-tips.html}
\end{toile}

\end{frame}

%%%%%%%%%%%%%%%%%%%%%%%%%%%%%%%%%%%%%%%%%%%%%%%%%%%%%%%%%%%%%%%%%%%%%%%%%%%%%%%%

\begin{frame}[fragile]{Tests de performances}

   TODO
   WIP

\begin{toile}
\toileurl{https://www.postgresql.org/docs/15/pgbench.html}
\end{toile}

\end{frame}

%%%%%%%%%%%%%%%%%%%%%%%%%%%%%%%%%%%%%%%%%%%%%%%%%%%%%%%%%%%%%%%%%%%%%%%%%%%%%%%%

\begin{frame}[fragile]{HOT - Heap-Only Tuples}

   TODO
   WIP

\begin{toile}
\toileurl{https://www.postgresql.org/docs/15/storage-hot.html}
\end{toile}

\end{frame}

