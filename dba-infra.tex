\documentclass[aspectratio=43]{beamer}
%\documentclass[aspectratio=1610]{beamer}
%\documentclass[aspectratio=169]{beamer}

% @IMP@ \usepackage{pgfpages}
% @IMP@ \pgfpagesuselayout{2 on 1}[a4paper,border shrink=5mm]

\usepackage{fixltx2e}

\usepackage[english,french]{babel}
\usepackage[T1]{fontenc}
\usepackage[utf8]{inputenc}
\usepackage{microtype}

\usepackage{lmodern}

\usepackage{amsmath}
\usepackage{array}
\usepackage{bookmark}
\usepackage{graphicx}
\usepackage{keystroke}
\usepackage{marvosym}
\usepackage{siunitx}
\usepackage{tabularx}
\usepackage{tabulary}
\usepackage{tikz}
\usepackage{xspace}

\usepackage{LPI}

% beamer

\usetheme[formation]{Linagora}

\title{LPIC-1}
\subtitle{Examen 101}

%\setbeameroption{show notes}

% array

\setlength{\extrarowheight}{2pt}

% babel

\frenchbsetup{og=«,fg=»}

\newcommand{\anglais}[1]{\textit{\foreignlanguage{english}{#1}}}

% graphicx

\graphicspath{{illustrations/}}

% marvosym

\newcommand{\curseur}{\Rectsteel}

% siunitx

\sisetup
{
	binary-units	= true ,
}

\DeclareSIUnit{\octet}{o}

% tikz

\usetikzlibrary{calc,decorations.pathmorphing}

% LPI

\LPIinit{101}

%%%%%%%%%%%%%%%%%%%%%%%%%%%%%%%%%%%%%%%%%%%%%%%%%%%%%%%%%%%%%%%%%%%%%%%%%%%%%%%%

\begin{document}

%%%%%%%%%%%%%%%%%%%%%%%%%%%%%%%%%%%%%%%%%%%%%%%%%%%%%%%%%%%%%%%%%%%%%%%%%%%%%%%%

\begin{frame}

\pdfbookmark[2]{\inserttitle\ -- \insertsubtitle}{titre}

\titlepage

\end{frame}

%%%%%%%%%%%%%%%%%%%%%%%%%%%%%%%%%%%%%%%%%%%%%%%%%%%%%%%%%%%%%%%%%%%%%%%%%%%%%%%%

\section*{Introduction}

\pdfbookmark[2]{Introduction}{introduction}

%%%%%%%%%%%%%%%%%%%%%%%%%%%%%%%%%%%%%%%%%%%%%%%%%%%%%%%%%%%%%%%%%%%%%%%%%%%%%%%%

\include{share/lpi-intro}
\include{lpic-1}
\include{share/lpi-examens}

%%%%%%%%%%%%%%%%%%%%%%%%%%%%%%%%%%%%%%%%%%%%%%%%%%%%%%%%%%%%%%%%%%%%%%%%%%%%%%%%

\begin{frame}{L'examen 101}

\begin{itemize}

\item Dure 1 h 30.

\item Comprend 60 questions.

\item Le score maximum est de 800.

\item Un score de 500 est nécessaire pour réussir l'examen.

\item C'est-à-dire 62,5 \% du score maximum soit environ 37,5 questions.

\end{itemize}

\begin{toile}
\toileurl{http://www.lpi.org/linux-certifications/programs/lpic-1/exam-101/}
\toileurl{http://wiki.lpi.org/wiki/LPIC-1_Objectives(FR)\#Objectifs_de_l.27examen_LPI_101}
\end{toile}

\end{frame}

%%%%%%%%%%%%%%%%%%%%%%%%%%%%%%%%%%%%%%%%%%%%%%%%%%%%%%%%%%%%%%%%%%%%%%%%%%%%%%%%

\include{share/precautions}

%%%%%%%%%%%%%%%%%%%%%%%%%%%%%%%%%%%%%%%%%%%%%%%%%%%%%%%%%%%%%%%%%%%%%%%%%%%%%%%%

\section*{Bibliographie}

\pdfbookmark[2]{Bibliographie}{bibliographie}

\include{bibliographie-101}
\include{bibliographie}
\include{share/bibliographie}

%%%%%%%%%%%%%%%%%%%%%%%%%%%%%%%%%%%%%%%%%%%%%%%%%%%%%%%%%%%%%%%%%%%%%%%%%%%%%%%%

\section*{Webographie}

\pdfbookmark[2]{Webographie}{webographie}

\include{webographie-101}

%%%%%%%%%%%%%%%%%%%%%%%%%%%%%%%%%%%%%%%%%%%%%%%%%%%%%%%%%%%%%%%%%%%%%%%%%%%%%%%%

\include{share/examens-blancs}

%%%%%%%%%%%%%%%%%%%%%%%%%%%%%%%%%%%%%%%%%%%%%%%%%%%%%%%%%%%%%%%%%%%%%%%%%%%%%%%%

\section*{Sommaire}

\pdfbookmark[2]{Sommaire}{sommaire}

\begin{frame}{Sommaire}

\tableofcontents[hideallsubsections]

\end{frame}

%%%%%%%%%%%%%%%%%%%%%%%%%%%%%%%%%%%%%%%%%%%%%%%%%%%%%%%%%%%%%%%%%%%%%%%%%%%%%%%%

\include{101}
\include{102}
\include{103}
\include{104}

%%%%%%%%%%%%%%%%%%%%%%%%%%%%%%%%%%%%%%%%%%%%%%%%%%%%%%%%%%%%%%%%%%%%%%%%%%%%%%%%

\section*{Conclusion}

\pdfbookmark[2]{Conclusion}{conclusion}

\merci

%%%%%%%%%%%%%%%%%%%%%%%%%%%%%%%%%%%%%%%%%%%%%%%%%%%%%%%%%%%%%%%%%%%%%%%%%%%%%%%%

\end{document}
