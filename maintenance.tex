%%%%%%%%%%%%%%%%%%%%%%%%%%%%%%%%%%%%%%%%%%%%%%%%%%%%%%%%%%%%%%%%%%%%%%%%%%%%%%%%

\section{Maintenance}

%%%%%%%%%%%%%%%%%%%%%%%%%%%%%%%%%%%%%%%%%%%%%%%%%%%%%%%%%%%%%%%%%%%%%%%%%%%%%%%%

\begin{frame}[fragile]{EXPLAIN PLAN}

   TODO
   WIP

\begin{toile}
\toileurl{XXXX}
\end{toile}

\end{frame}

%%%%%%%%%%%%%%%%%%%%%%%%%%%%%%%%%%%%%%%%%%%%%%%%%%%%%%%%%%%%%%%%%%%%%%%%%%%%%%%%

\begin{frame}[fragile]{Tâches de maintenance de la base de données}

   La base de données nécessite des opérations de maintenance régulière pour assurer un service optimal.
   Les principales tâches de maintenance sont:
   \begin{itemize}
      \item Les sauvegardes régulières qui permettront de s'en sortir en cas d'accident grave
      \item Les opérations de VACUUM
      \item Les mises à jour des statistiques
      \item Les mises à jour des indexes
      \item La gestion des logs
   \end{itemize}

\begin{toile}
\toileurl{https://www.postgresql.org/docs/15/maintenance.html}
\end{toile}

\end{frame}

%%%%%%%%%%%%%%%%%%%%%%%%%%%%%%%%%%%%%%%%%%%%%%%%%%%%%%%%%%%%%%%%%%%%%%%%%%%%%%%%

\begin{frame}[fragile]{Les opérations de VACUUM}

   Les opérations de VACUUM sont lancées automatiquement par le démon \textsf{autovacuum}. Certains DBAs préfèrent gérer ce process eux-même à partir de tâches de type cron.
   Les raisons principales pour lancer un VACUUM régulier sont:

   \begin{itemize}
      \item récupérer l'espace disque occupée pour des lignes mises à jour ou supprimées
      \item mises à jour des statistiques utilisées par le planificateur de requêtes (query planner)
      \item mises à jour du flag de visibilité pour améliorer la performance des indexes only scan (chapitre Optimisation)
      \item protection contre la perte de données très anciennes causée par la rotation des identifiants de transaction ou de multi-transaction
   \end{itemize}

\begin{toile}
\toileurl{https://www.postgresql.org/docs/15/routine-vacuuming.html}
\end{toile}

\end{frame}

%%%%%%%%%%%%%%%%%%%%%%%%%%%%%%%%%%%%%%%%%%%%%%%%%%%%%%%%%%%%%%%%%%%%%%%%%%%%%%%%

\begin{frame}{Différents type de VACUUM}

   Il existe 2 types de VACUUM:
   \begin{itemize}
      \item le VACUUM standard
      \item le VACUUM FULL 
   \end{itemize}

\end{frame}

%%%%%%%%%%%%%%%%%%%%%%%%%%%%%%%%%%%%%%%%%%%%%%%%%%%%%%%%%%%%%%%%%%%%%%%%%%%%%%%%

\begin{frame}{VACUUM standard}

   \begin{itemize}
      \item Le VACUUM standard peut être lancé en parallèle des commandes SQL (\textbf{SELECT}, \textbf{UPDATE}, \textbf{INSERT} et \textbf{DELETE})
      \item la commande \textbf{ALTER TABLE} ne peut être exécutée sur une table en cours de traitement par le process
      \item le VACUUM génère des I/O très importantes qui diminuent les performances des autres sessions actives
      \item Il est possible de tempérer la charge I/O causée par le VACUUM par l'intermédiaire des paramètres: \textbf{vacuum\_cost\_*},
   \end{itemize}

\end{frame}

%%%%%%%%%%%%%%%%%%%%%%%%%%%%%%%%%%%%%%%%%%%%%%%%%%%%%%%%%%%%%%%%%%%%%%%%%%%%%%%%

\begin{frame}{Les paramètres \textbf{vacuum\_cost\_*}}

   \begin{itemize}
      \item Le VACUUM standard peut être lancé en parallèle des commandes SQL (\textbf{SELECT}, \textbf{UPDATE}, \textbf{INSERT} et \textbf{DELETE})
      \item la commande \textbf{ALTER TABLE} ne peut être exécutée sur une table en cours de traitement par le process
      \item le VACUUM génère des I/O très importantes qui diminuent les performances des autres sessions actives
      \item Il est possible de tempérer la charge I/O causée par le VACUUM par l'intermédiaire des paramètres: \textbf{vacuum\_cost\_*},
   \end{itemize}

\begin{toile}
\toileurl{https://www.postgresql.org/docs/15/runtime-config-resource.html\#RUNTIME-CONFIG-RESOURCE-VACUUM-COST}
\end{toile}
   
\end{frame}

%%%%%%%%%%%%%%%%%%%%%%%%%%%%%%%%%%%%%%%%%%%%%%%%%%%%%%%%%%%%%%%%%%%%%%%%%%%%%%%%

\begin{frame}{VACUUM FULL}

   \begin{itemize}
      \item Ce VACUUM recupère plus d'espace disque que le standard.
      \item Il est cependant plus lent
   \end{itemize}

\end{frame}

%%%%%%%%%%%%%%%%%%%%%%%%%%%%%%%%%%%%%%%%%%%%%%%%%%%%%%%%%%%%%%%%%%%%%%%%%%%%%%%%

\begin{frame}[fragile]{Tests de vérification de la bonne santé de la base}

   TODO
   WIP

\begin{toile}
\toileurl{https://bucardo.org/check\_postgres/}
\end{toile}

\end{frame}

%%%%%%%%%%%%%%%%%%%%%%%%%%%%%%%%%%%%%%%%%%%%%%%%%%%%%%%%%%%%%%%%%%%%%%%%%%%%%%%%
