%%%%%%%%%%%%%%%%%%%%%%%%%%%%%%%%%%%%%%%%%%%%%%%%%%%%%%%%%%%%%%%%%%%%%%%%%%%%%%%%

\section{Supervision du serveur de base de données}

%%%%%%%%%%%%%%%%%%%%%%%%%%%%%%%%%%%%%%%%%%%%%%%%%%%%%%%%%%%%%%%%%%%%%%%%%%%%%%%%

\begin{frame}[fragile]{Supervision de la base de données}

   \begin{itemize}
      \item En premier lieu, les outils Unix suivants sont très utiles:
      \begin{itemize}
         \item ps: liste les process Unix, leur état, leur stockage en mémoire
         \item top: classe les process en fonction de différents critères et donne une idée de la charge du serveur
         \item iostat: liste les statistiques des I/O du serveur
         \item vmstat: remote des statistiques sur la mémoire virtuelle du serveur
      \end{itemize}
   \end{itemize}

\begin{toile}
\toileurl{https://www.postgresql.org/docs/15/monitoring.html}
\end{toile}

\end{frame}

%%%%%%%%%%%%%%%%%%%%%%%%%%%%%%%%%%%%%%%%%%%%%%%%%%%%%%%%%%%%%%%%%%%%%%%%%%%%%%%%

\begin{frame}[fragile]{Le système de statistiques cumulatif}

   \begin{itemize}
      \item Le système de statistiques cumulatif de PostgreSQL collecte et informe le DBA de l'activité du serveur
      \item Actuellement, l'accès aux tables et aux indexes - que ce soit au niveau ligne ou au niveau bloc - sont comptabilisés
      \item Le nombre total de lignes de chaque table et les informations sur le VACUUM et l'ANALYZE sont inclus
      \item Dans le cas où l'option est activée, les appels aux fonctions définies par l'utilisateur et leur temps d'utilisation sont également inclus dans les statistiques
   \end{itemize}

\begin{toile}
\toileurl{https://www.postgresql.org/docs/15/monitoring-stats.html}
\end{toile}

\end{frame}

%%%%%%%%%%%%%%%%%%%%%%%%%%%%%%%%%%%%%%%%%%%%%%%%%%%%%%%%%%%%%%%%%%%%%%%%%%%%%%%%

\begin{frame}[fragile]{Les statistiques dynamiques}

   \begin{itemize}
      \item PostgreSQL informe également le DBA des événements actuellement en cours sur la base de données
      \item Par exemple, la liste des requêtes en cours sur le serveur
      \item La liste des connexions en cours sur le serveur
      \item Cette fonctionnalité est indépendante du système de statistiques cumulatif
   \end{itemize}

\end{frame}

%%%%%%%%%%%%%%%%%%%%%%%%%%%%%%%%%%%%%%%%%%%%%%%%%%%%%%%%%%%%%%%%%%%%%%%%%%%%%%%%

\begin{frame}[fragile]{Activation des statistiques cumulatives - \textbf{track\_*}}

   \begin{itemize}
      \item La collecte de statistiques engendre une activité supplémentaires, le système permet au DBA d'activer ou non la collecte
      \item Les paramètres d'activation de la collecte sont dans postgresql.conf
      \item \textbf{track\_activities}: supervision de la commande en cours exécutée sur le serveur
      \item \textbf{track\_counts}: active le système de collecte de statistiques sur les tables et les indexes
      \item \textbf{track\_counts}: surveillance de l'utilisation des fonctions utilisateurs
      \item \textbf{track\_io\_timing}: supervision des temps de lecture/écrire de blocs
      \item \textbf{track\_wal\_io\_timing}: supervision des temps d'écrire de WAL
   \end{itemize}

\begin{toile}
\toileurl{https://www.postgresql.org/docs/15/runtime-config-statistics.html}
\end{toile}

\end{frame}

%%%%%%%%%%%%%%%%%%%%%%%%%%%%%%%%%%%%%%%%%%%%%%%%%%%%%%%%%%%%%%%%%%%%%%%%%%%%%%%%

\begin{frame}[fragile]{Activation globale ou par session}

   \begin{itemize}
      \item Les paramètres \textbf{track\_*} enregistrés dans postgresql.conf s'appliquent à l'ensemble des process du serveur
      \item Il est possible de les activer/désactiver par session avec la commande \textbf{SET}
      \item Par mesure de sécurité, seuls les super-utilisateurs peuvent utiliser la commande SET
   \end{itemize}

\begin{toile}
\toileurl{https://www.postgresql.org/docs/15/sql-set.html}
\end{toile}

\end{frame}

%%%%%%%%%%%%%%%%%%%%%%%%%%%%%%%%%%%%%%%%%%%%%%%%%%%%%%%%%%%%%%%%%%%%%%%%%%%%%%%%

\begin{frame}[fragile]{Stockage des statistiques collectées}

   \begin{itemize}
      \item Les statistiques cumumlées sont stockées dans la mémoire partagée (\textbf{shared\_buffers})
      \item Chaque process collecte les informations locales puis met à jour la mémoire partagée à intervalles réguliers
      \item Lorsqu'un serveur (les réplicats inclus) s'arrête proprement, une copie permanente des statistiques est stockée dans le répertoire \textbf{pg\_stat}
      \item Cela permet de récupérer les statistiques au prochain démarrage du serveur
      \item En cas d'arrêt brutal du serveur (arrêt immédiat, crash, redémarrage à partir d'un basebackup ou d'un PITR), ces statistiques sont remis à zéro
   \end{itemize}

\end{frame}

%%%%%%%%%%%%%%%%%%%%%%%%%%%%%%%%%%%%%%%%%%%%%%%%%%%%%%%%%%%%%%%%%%%%%%%%%%%%%%%%

\begin{frame}[fragile]{Lecture des statistiques collectées}

   \begin{itemize}
      \item Il existe différentes vues pour accéder aux statistiques collectées
      \item Ces vues ne sont pas mises instantanément à jour
      \item Chaque process met à jour ces vues à un intervalle max de \textbf{PGSTAT\_MIN\_INTERVAL} millisecondes (valeur par défaut 1 seconde)
      \item Cependant, les statistiques collectées par \textbf{track\_activities} sont à jour
      \item Lorsqu'un process fournit des statistiques, les valeurs accédées sont mises en cache jusqu'à la fin de la transaction
      \item Cela signifie que ces informations resteront inchangées tout au long de la transaction
      \item Ce n'est pas un bug mais une feature :) dans le sens, où l'ensemble des statistiques restent cohérents pendant la transaction
      \item Cela permet d'avoir une vue cohérente pour les outils de collecte de statistiques
   \end{itemize}

\end{frame}

%%%%%%%%%%%%%%%%%%%%%%%%%%%%%%%%%%%%%%%%%%%%%%%%%%%%%%%%%%%%%%%%%%%%%%%%%%%%%%%%

\begin{frame}[fragile]{Décalage des statistiques collectées - \textbf{stats\_fetch\_consistency}}

   \begin{itemize}
      \item Lorsque les statistiques sont récupérées de manière interactive ou en passant par des requêtes coûteuses, l'accès aux différents résultats peut engendrer un retard important dans l'affichage des résultats
      \item Pour minimiser le retard, le paramètre \textbf{stats\_fetch\_consistency} peut être positionné avec la valeur \textit{snapshot}
      \item Cela engendre une utilisation plus importante de mémoire partagée
      \item Dans le cas où l'accès aux statistiques n'est pas fréquent, ce paramètre peut être positionné à \textit{none}
      \item La fonction \textbf{pg\_stat\_clear\_snapshot()} peut être utilisée pour supprimer le snapshot de statistiques
      \item L'accès aux statistiques est autorisé pour les super-utilisateurs ou pour le rôle ayant la permission \textbf{pg\_read\_all\_stats}
   \end{itemize}

\end{frame}

%%%%%%%%%%%%%%%%%%%%%%%%%%%%%%%%%%%%%%%%%%%%%%%%%%%%%%%%%%%%%%%%%%%%%%%%%%%%%%%%

\begin{frame}[fragile]{Statistiques d'une transaction - Vues \textbf{pg\_stat\_xact*}}

   \begin{itemize}
      \item Une transaction peut voir ses propres statistiques en temps réel non encore écrits en cache
      \item Ces statistiques sont stockées dans les vues:
      \begin{itemize}
         \item pg\_stat\_xact\_all\_tables
         \item pg\_stat\_xact\_sys\_tables
         \item pg\_stat\_xact\_user\_tables
         \item pg\_stat\_xact\_user\_functions
      \end{itemize}
   \end{itemize}

\end{frame}

%%%%%%%%%%%%%%%%%%%%%%%%%%%%%%%%%%%%%%%%%%%%%%%%%%%%%%%%%%%%%%%%%%%%%%%%%%%%%%%%

\begin{frame}[fragile]{Liste des vues}

   La liste des vues pour les statistiques dynamiques est:
   \begin{itemize}
      \item pg\_stat\_activity
      \item pg\_stat\_replication
      \item pg\_stat\_wal\_receiver
      \item pg\_stat\_recovery\_prefetch
      \item pg\_stat\_subscription
      \item pg\_stat\_ssl
      \item pg\_stat\_gssapi
   \end{itemize}

\end{frame}

%%%%%%%%%%%%%%%%%%%%%%%%%%%%%%%%%%%%%%%%%%%%%%%%%%%%%%%%%%%%%%%%%%%%%%%%%%%%%%%%

\begin{frame}[fragile]{Liste des vues}

   \begin{itemize}
      \item pg\_stat\_progress\_analyze
      \item pg\_stat\_progress\_create\_index
      \item pg\_stat\_progress\_vacuum
      \item pg\_stat\_progress\_cluster
      \item pg\_stat\_progress\_basebackup
      \item pg\_stat\_progress\_copy
   \end{itemize}

\end{frame}

%%%%%%%%%%%%%%%%%%%%%%%%%%%%%%%%%%%%%%%%%%%%%%%%%%%%%%%%%%%%%%%%%%%%%%%%%%%%%%%%%
%
%\begin{frame}[fragile]{Tests de vérification de la bonne santé de la base}
%
%   TODO
%   WIP
%
%\begin{toile}
%\toileurl{https://bucardo.org/check\_postgres/}
%\end{toile}
%
%\end{frame}
%
%%%%%%%%%%%%%%%%%%%%%%%%%%%%%%%%%%%%%%%%%%%%%%%%%%%%%%%%%%%%%%%%%%%%%%%%%%%%%%%%

\begin{frame}{Supervision de l'activité des disques}

\begin{itemize}

   \item Chaque table possède un fichier primaire en disque où la majorité de ses données sont stockées
   \item Ce fichier est appelé \textbf{heap} (tas)
   \item Dans le cas où la table possède des colonnes ayant une valeur très large, un fichier de type \textbf{TOAST} est associé à la table
   \item Il peut y avoir également des indexes associés à la table
   \item Ces indexes sont stockés à un endroit différent du disque
   \item Chaque table et index est stocké sur un fichier séparé. Il est possible que ce soit plusieurs fichiers dans le cas où la taille dépasse 1 Goctets

\end{itemize}

\begin{toile}
\toileurl{https://www.postgresql.org/docs/15/diskusage.html}
\end{toile}

\end{frame}

%%%%%%%%%%%%%%%%%%%%%%%%%%%%%%%%%%%%%%%%%%%%%%%%%%%%%%%%%%%%%%%%%%%%%%%%%%%%%%%%

\begin{frame}{Outils de supervision des disques}

\begin{itemize}

   \item Il existe 3 méthodes de supervision de l'utilisation des disques:
   \begin{itemize}
      \item les fonctions SQL d'administration
      \item le module \textbf{oid2name}
      \item une inspection manuelle du catalogue système
   \end{itemize}
\item La $1^{ère}$ méthode est la plus recommandée

\end{itemize}

\begin{toile}
\toileurl{https://www.postgresql.org/docs/15/functions-admin.html\#FUNCTIONS-ADMIN-DBSIZE}
\toileurl{https://www.postgresql.org/docs/15/oid2name.html}
\end{toile}

\end{frame}

%%%%%%%%%%%%%%%%%%%%%%%%%%%%%%%%%%%%%%%%%%%%%%%%%%%%%%%%%%%%%%%%%%%%%%%%%%%%%%%%
